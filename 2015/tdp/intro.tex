\section{Introduction}
\label{sec:intro}

Rescue operations in disaster situations, such as earthquakes, tsunamis, avalanches, for example, are a serious social issue. These operations involve different agents (police forces, firemen, paramedics, among others) in a hostile environment: buildings collapse, civilians get wounded, streets get blocked, water and power supplies get compromised and communication is limited, making information about the problem scarce and imprecise.

Such situations demand systems that can create robust, dynamic and intelligent search and rescue plans to aid human effort. Within this context, the goal in RoboCup Rescue Agent \& Multi-agent challenge competition is to foster research and development in the field of coordination in multiagent systems. In 2015, the challenge is to program teams of fire brigades, paramedic and police forces in the Agent RoboCup Rescue platform \citep{Kitano2000} and a decentralized constraint optimization problem (DCOP) algorithm in RMASBench \citep{Kleiner+2013}. 

% must coordinate in order to minimize the damage caused in a city by an earthquake.

This paper describes the strategies that will be used to guide \teamname team in  Robocup Rescue Agent \& Multi-agent challenge competition. It is divided in two main parts. In the first part (Section \ref{sec:agent}), we describe our main goals in the Agent competition, which are: to develop a model to predict how the fire evolves in burning buildings, to define a behaviour-based strategy for the police force agents, to use an efficient message compression scheme to save bandwidth in communication and to create a simple auction-based method for recruiting in tasks that demand more than one agent. In the second part (Section \ref{sec:multi-agent}), we describe our approach to the Multi-agent competition, which involves the implementation of an scalable algorithm based on swarm intelligence for task allocation in dynamic environments, called eXtreme-Ants \citep{Santos&Bazzan2009optmas}.
