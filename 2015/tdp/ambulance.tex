\subsection{Ambulance Team}
\label{sec:ambulance}
% O mapa será setorizado para as ambulancias também. Número de ambulancias em cada setor dependerá da importância do mesmo.
% Será feita uma regressão para calcular a estimativa de vida do civil. A prioridade do resgate de um civil será dada por um cálculo envolvendo sua estimativa de vida e a distância até ele.
% o principal é a estimativa de tempo de vida de um soterrado, dado seu damage, buriedness e hp atual.

This team is responsible by the most important task in a disaster environment, namely the search and rescue of injured civilians. Seeking for this goal, the ambulance team will be aided by the other agents with respect to the discovery of new victims.
% Esta equipe é responsável pela atividade de maior importância em um ambiente de desastre como o descrito na simulação, o de busca e resgate de civis feridos. Visando este objetivo, esta equipe é auxiliada pelas demais no que se diz respeito ao descobrimento de alvos de resgate de modo que a rede de comunicação faça com que nenhum civil seja por ventura, deixado de ser salvo.

Searching for efficiency in rescue, the team of paramedics will be splitted into sectors, defined by the number of available agents and map size. These regions will have a category of importance measured by the quantity of civilians not rescued yet, ordering the contingent of agents to be designated to most important areas. Separating the agents into regions of interest will make them cover all the map and save, in thesis, the majority of civilians.
% Buscando pela eficiência no resgate, os tipos de paramédicos serão divididos em setores, definidos pela quantidade de agentes deste tipo disponível. Estas regiões serão regidas por um grau de importância medida pela quantidade de civis ainda por serem resgatados naquela área. Isso fará com que o contingente de agentes seja designado para áreas mais interessantes para realização de resgates.

\subsubsection{Task Allocation}

The task prioritization of this team will be governed by the life estimation of the alive civilians. This estimation takes into account the following:

\begin{itemize}
\item Distance between the agent and the civilian;
\item The civilian's life;
\item The civilian's damage;
\item If the civilian is inside or not a burning building;
\item If exists some blockage between the agent and the civilian.
\end{itemize}

Based on these values the agents will tend to save civilians with bigger probability of survival, lowering the priority of the others civilians with less chance to live.

%upon the circumstances of scenario.
%the life estimation of all targets of rescue. This estimation will take into account the distance between the agent and the task, as well as your life and severity of your injury, if the civilian is inside or not a burning building or if exists some blockage between them. With this values, the agents must incline to civilians with bigger probability of survive. upon the circunstances of scenario.
% A priorização das tarefas dentre a equipe de ambulancias será regida pela estimativa de vida de todos os alvos de resgate. Esta estimativa levará em consideração a distância entre o agente e o mesmo, bem como sua vida e a severidade do ferimento do mesmo, além de se o mesmo se encontra ou não dentro de construções em chamas ou se existem bloqueios entre o caminho do agente e do alvo. Dados estes valores, os agentes devem se inclinar para os civis que tem maior probabilidade de sobrevivencia, mediante as circunstancias do cenário.
