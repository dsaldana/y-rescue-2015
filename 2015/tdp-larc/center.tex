\subsection{Center agents}
\label{sec:center}
In our strategy, center agents listen to radio communication between agents, accumulating knowledge contained in the messages and replicating important messages to avoid loss of information due to communication unreliability. Using the accumulated knowledge from listening to multiple communication channels, center agents are able to build a more accurate world view. Eventually, center agents send messages to agents in specific channels to give them information previously sent in other channels. This is done in order to make knowledge among agents more uniform.

Center agents do not assign tasks to field agents. Their role is merely informational. The decision of engaging in a task is made individually by each field agent. Therefore, our field agents are autonomous decision-makers. The advantage of this approach is that our strategy does not depend heavily on the center agents. Although their knowledge bases will lack some information provided by the center agents, the decision-making process of our field agents will not change, thus they should not be heavily affected by the absence of center agents.

%Nossa abordagem nao depende fortemente de centrais; elas serao usadas apenas para escutar mais de um canal e replicar/propagar mensagens ``importantes''
