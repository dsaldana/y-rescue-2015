\section{Conclusion}

To address the challenge posed by the agent competition, we present an approach that relies on autonomous decision-making agents. This means that no field or center agent can directly assign a task to other agent. %Each agent decides the task it will engage based on its current knowledge, accumulated from observation and communication. Cooperation emerges when agents decide to engage in the same task, or when help is requested via a recruitment method based which is a simplification eXtreme-Ants recruitment process. Even during recruitment, agents are not directly recruited: they can decide whether to offer help to the recruiter or not. The role of center agents is informational, thus in their absence, the decision-making process of field agents does not change.

Our approach relies on map sectorization in order to limit the scope of action of each field agent. The task allocation of field agents is based on the score that they assign to each task, which is associated with the importance and difficulty level of solving the task. %We believe that the proposed approach is flexible and robust. Flexibility comes from the fact that the task allocation process for all agents is similar: they assign scores to the tasks that they know and engage in the task with the highest score. Robustness comes from the fact that agents do not depend on each other to decide which task to engage. Thus, the approach can work for several configurations of team sizes and presence of center agents.

%For the multi-agent competition, we choose to implement eXtreme-Ants in RMASBench. The algorithm compares favorably to other E-GAP based approaches in terms of team reward, computational effort and exchanged messages. This implementation represents a scientific contribution as we can establish the first comparison of E-GAP and factor graph based approaches (e.g. max-sum already implemented in RMASBench) for task allocation in dynamic environments.% in a single framework.
%Also, it would be interesting to compare its performance with factor-graph based algorithms, such as max-sum, which is implemented in RMASBench \cite{Kleiner+2013}. For further details of eXtreme-Ants implementation and performance, the reader can refer to \cite{Santos&Bazzan2009optmas}.

%falar das estrategias gerais; resultados esperados
% em suma, nossos agentes sao 'autonomos', isto eh, nao recebem ordem direta de outros agentes, mas deliberam por conta propria se vao realizar as tarefas ou nao