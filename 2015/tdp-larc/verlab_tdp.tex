\documentclass[a4paper, 10pt, conference]{ieeeconf}      % Use this line for a4 paper

\IEEEoverridecommandlockouts                              % This command is only needed if 
                                                          % you want to use the \thanks command

\overrideIEEEmargins                                      % Needed to meet printer requirements.

%\usepackage[square,sort,comma,numbers]{natbib}
%\usepackage[style=numeric-comp]{biblatex}
\usepackage[english]{babel}
\usepackage[utf8]{inputenc}
\usepackage{amsmath}
\usepackage{algorithm}
\usepackage{algorithmic}
\usepackage{graphicx}
\usepackage{subfigure}
\usepackage{url}

%\bibliography{references}

\newcommand{\teamname}{Y-Rescue~}

\hyphenation{av-a-lanches}

\title{\teamname Team Description Paper}

\author{
	Anderson Tavares$^{*1}$, Alex Barros$^{1}$, Elerson Rubens$^{1}$, Gabriel Lucas$^{1}$, \\ Hector Azpúrua$^{1}$, 
		Jhielson Pimentel$^{1}$, Lucas Batista$^{1}$, Luiz Emery$^{1}$, Omar Pino$^{1}$, \\
		Phelipe Andrade$^{1}$, Rafael Colares$^{1}$, Douglas Macharet$^{1}$, Luiz Chaimowicz$^{1}$ 
		\thanks{Authors would like to thank FAPEMIG for supporting the team through its ``Santos Dummont'' program and to CAPES and CNPq for the researchers' fellowships.}
		\thanks{$^{1}$VeRLab -- Departamento de Ciência da Computação, Universidade Federal de Minas Gerais, Brazil}
		\thanks{$^{*}$Corresponding author: {\tt\small anderson@dcc.ufmg.br}}
}



%\thanks{*This work was not supported by any organization}% <-this % stops a space

\newcommand{\agentset}{\mathcal{I}}
\newcommand{\taskset}{\mathcal{J}}

\newcommand{\agtres}[1]{\ensuremath{r_{#1}}}
\newcommand{\agtcap}[2]{\ensuremath{k_{#1#2}}}
\newcommand{\consumes}[2]{\ensuremath{c_{#1#2}}}
\newcommand{\allocate}[2]{\ensuremath{a_{#1#2}}}
\newcommand{\partialrwd}[2]{\ensuremath{w_{#1#2}}}
\newcommand{\andtasks}[1]{\ensuremath{\alpha_{#1}}}
\newcommand{\allandtasks}{\ensuremath{\bowtie}}

\newcommand{\msg}[1]{\textit{#1}}
\newcommand{\behavior}[1]{\textit{#1}}
%\newcommand{\egminus}{E-GAP$^-$}

\newcommand{\stimulus}[1]{\ensuremath{s_#1}}
\newcommand{\respthresh}[2]{\ensuremath{\theta_{#1#2}}}
\newcommand{\tendency}[2]{\ensuremath{T_{#1#2}}}

\newcommand{\variables}{\ensuremath{\mathcal{X}}}
\newcommand{\domains}{\ensuremath{\mathcal{D}}}
\newcommand{\functions}{\ensuremath{\mathcal{F}}}

% Black magic xD
%\makeatletter
%\renewenvironment{thebibliography}[1]
%     {\section*{\bibname}%
%      \@mkboth{\MakeUppercase\bibname}{\MakeUppercase\bibname}%
%      \list{\@biblabel{\@arabic\c@enumiv}}%
%           {\settowidth\labelwidth{\@biblabel{#1}}%
%            \leftmargin\labelwidth
%            \advance\leftmargin\labelsep
%            \@openbib@code
%            \usecounter{enumiv}%
%            \let\p@enumiv\@empty
%            \renewcommand\theenumiv{\@arabic\c@enumiv}}%
%      \sloppy
%      \clubpenalty4000
%      \@clubpenalty \clubpenalty
%      \widowpenalty4000%
%      \sfcode`\.\@m}
%     {\def\@noitemerr
%       {\@latex@warning{Empty `thebibliography' environment}}%
%      \endlist}
%\makeatother

\begin{document}


\maketitle

\begin{abstract}
This paper describes the strategy of \teamname team to the LARC RoboCup Rescue 2015 Agent competition. %The importance of this type of simulation is to test and improve plans of rescue in disaster environments. 
In our approach, center agents are relegated to an informational role whereas field agents are capable of individual decision making. Our main contributions are the path planning via Dijkstra's algorithm in an augmented neighborhood graph and the clearing technique adopted by the police forces. %and the exploration scheme adopted by the ambulance team. 
Effort estimation for task performance is based on simulations that use the Rescue Simulator engine where unknown parameters are replaced with estimated values. Items such as the behaviour-based strategy for police forces and the simple recruiting mechanism in tasks that demand more than one agent are similar to the ones used in Hefei's 2015 RoboCup Rescue Agent competition.
\end{abstract}

%%% SECTION: Introduction
\section{Introduction}
\label{sec:intro}

Rescue operations in disaster situations, such as earthquakes, tsunamis, avalanches, for example, are a serious social issue. These operations involve different agents (police forces, firefighters, paramedics, among others) in a hostile environment: buildings collapse, civilians get wounded, streets get blocked, water and power supplies get compromised and communication is limited, making information about the problem scarce and imprecise.

Such situations demand systems that can create robust, dynamic and intelligent search and rescue plans to aid human effort. Within this context, the goal in RoboCup Rescue Agent competition is to foster research and development in the field of coordination in multiagent systems. In 2015, the challenge is to program teams of fire brigades, paramedic and police forces in the Agent RoboCup Rescue platform \cite{Kitano2000}.% and a decentralized constraint optimization problem (DCOP) algorithm in RMASBench \cite{Kleiner+2013}. 

% must coordinate in order to minimize the damage caused in a city by an earthquake.

This paper describes the strategies that will be used to guide \teamname team in  LARC Rescue Agent competition focusing on our main contributions, which are: path planning using Dijkstra algorithm in an augmented neighborhood graph (Section \ref{sec:path}) and the clearing technique adopted by Police Forces (Section \ref{sec:police}). %and the map exploration strategy adopted by the ambulance team (Section \ref{sec:ambulance}).
%to develop a model to predict how the fire evolves in burning buildings, to define a behaviour-based strategy for the police force agents %to use an efficient message compression scheme to save bandwidth in communication 
%and to use a simple  method for recruiting in tasks that demand more than one agent. %In the second part (Section \ref{sec:multi-agent}), we describe our approach to the multi-agent competition, which involves the implementation of eXtreme-Ants \cite{Santos&Bazzan2009optmas}, a scalable algorithm based on swarm intelligence for task allocation in dynamic environments.


%\section*{Agent challenge competition}

%\section{Agent challenge competition}
%\label{sec:agent}

%%%SUBSECTION: Ambulance team
\section{Ambulance Team}
\label{sec:ambulance}

This team is responsible by the most important task in a disaster environment, namely the search and rescue of injured civilians. Seeking for this goal, the ambulance team will be aided by the other agents with respect to the discovery of new victims.

The team of paramedics will be divided into sectors or clusters, defined by the number of available agents and map size. Every agent will explore their designated region before engage any task, to be able to create at least a minimum map of their local region. 

The agents can visit the regions of other ambulances if in their specific region there is no new tasks or if another agent has asked for support. Every region or cluster will have a category of importance measured by the quantity of civilians not rescued yet, ordering the contingent of agents to be designated to most important areas. Separating the agents into regions of interest will make them cover all the map and save, in thesis, the majority of civilians.

% Buscando pela eficiência no resgate, os tipos de paramédicos serão divididos em setores, definidos pela quantidade de agentes deste tipo disponível. Estas regiões serão regidas por um grau de importância medida pela quantidade de civis ainda por serem resgatados naquela área. Isso fará com que o contingente de agentes seja designado para áreas mais interessantes para realização de resgates.

\subsection{Task Allocation}

The task prioritization of this team will be governed by the life estimation of the alive civilians. This estimation takes four factors into account: agent-civilian distance; civilian's life; civilian's damage and whether or not the civilian is buried or in a burning building.

The agents will tend to save the closer civilians with most chances of a successful rescue and survival, lowering the priority of the others civilians with less chance to live, less probability of be reached or to be rescued.

As ambulances can die from fire and damage, this agents need to be particularly careful with burning buildings. If the agent if rescuing a civilian and the building start catching fire, the agent will communicate the fire and search for the next best available civilian.

%Based on these features, the agents will tend to save civilians with higher probability of survival. %, lowering the priority of the others civilians with less chance to live.

%upon the circumstances of scenario.
%the life estimation of all targets of rescue. This estimation will take into account the distance between the agent and the task, as well as your life and severity of your injury, if the civilian is inside or not a burning building or if exists some blockage between them. With this values, the agents must incline to civilians with bigger probability of survive. upon the circunstances of scenario.
% A priorização das tarefas dentre a equipe de ambulancias será regida pela estimativa de vida de todos os alvos de resgate. Esta estimativa levará em consideração a distância entre o agente e o mesmo, bem como sua vida e a severidade do ferimento do mesmo, além de se o mesmo se encontra ou não dentro de construções em chamas ou se existem bloqueios entre o caminho do agente e do alvo. Dados estes valores, os agentes devem se inclinar para os civis que tem maior probabilidade de sobrevivencia, mediante as circunstancias do cenário.

\subsection{Rescuing}

The rescuing process takes a long time for one agent, due the fact that only one unit of buriedness can be removed at a time step. To speedup the rescuing process a recruitment protocol was implemented to allow agents to recruit another agents for cooperation in several tasks. In the ambulance, the cooperation allows other agents help with the unbury, improving the overall number of rescuing civilians.


%%%SUBSECTION: Police force
\subsection{Police Force}
\label{sec:police}
This team has two main goals: to clear obstructed paths and to scout unvisited areas of the map. The first goal can be divided into three jobs with different priorities, in order: i) clear obstructed paths of important buildings, ii) clear paths that are obstructing other agents, iii) clear random obstructed paths. The second goal is to continuously search the map for survivors and fire spots.

To achieve these goals, police force agents use a behaviour-based controller. Each agent chooses a behaviour based on its position and the location of important features on the map. The behaviours are:

\begin{itemize}

  \item \behavior{Sweeper}: the \textit{sweeper} behaviour will try to clear important paths. These paths includes the vicinity of important map features;

  \item \behavior{By-demand}: the \textit{by-demand} behaviour will clear their own vicinity and will constantly listen to the communication channels for help requests;

  \item \behavior{Scout}: the \textit{scout} behaviour will keep searching inside buildings for survivors.

\end{itemize}

\subsubsection{Task Allocation}
Initially, the map will be divided into \emph{n} clusters, where \emph{n} is the number of police force team agents. These clusters will be created taking into account the size of the map and the localization of each agent. Agents who are near an important feature of the map will be given the \behavior{sweeper} behaviour, while agents who are on other areas will be given the \behavior{by-demand} behaviour. When an agent finishes its tasks, it will then be assigned to the \behavior{scout} behaviour.

The \behavior{sweeper} agent task is simple and direct: it will clear its cluster's paths until all paths are clear. It will not listen to any communication channel or stop its task for any reason. The \behavior{by-demand} agent task is more complicated, as it will be listening to a communication channel and responding  to calls for help. An agent from any other team will be able to broadcast a distress message to the police force team informing that it is trapped and need assistance. This message contains the sender's current task and location. When a \behavior{by-demand} agent receives a help message and it is the closest to the sender agent, this \behavior{by-demand} agent will drop its current task and will respond to that message.

Both behaviours, \behavior{sweeper} and \behavior{by-demand}, will keep listening to calls for help from civilians and will keep searching for fire spots, in order to report these events to the proper agent team. Once they finish their tasks, they will be assigned to the \behavior{scout} behaviour. Agents with this behaviour will wander through the map helping other police force team agents and searching for survivors and fire spots.
% Inicialmente, existirão apenas fixos e sob demanda. Quando um fixo terminar sua tarefa, ele se torna sob-demanda. Ao longo do tempo, os sob demanda viram batedores.
% Implementar o Q-learning no agente policia
% dividi-los em dois times, um fica responsavel por desbloquear caminhos 'vitais' (até refugio ou hidrantes, por ex.) e outro time escuta os canais dos outros agentes para limpar o caminho deles até suas tarefas.

%\subsubsection{Task Allocation}


%%%SUBSECTION: Fire brigade
\subsection{Fire brigade}
\label{sec:firefighters}
Similarly to the ambulance team and police force, fire brigade agents will be divided in sectors of the map. The number of sectors depends on map size and number of fire brigades on the whole map. A score will be assigned to each sector, depending on its importance. For the fire brigades, the importance of a sector depends on the number of burning buildings and the disaster potential of the sector. The disaster potential is related to buried civilians, gas stations and building density of a sector. Building density takes into account the total area (ground area * number of floors) of all buildings in a sector. We want to prevent the fire from spreading to buildings with buried civilians, to gas stations and to areas with high building density.

Fire brigade agents will assign a score to each burning building depending on the ``effort'' needed to extinguish the fire on the building. The effort is related to the amount of water and time required to control the fire on a building. To calculate the effort, we need a model of how fire evolves in a building along the time. This model will be given by a regression over the attributes of the building that the fire brigade agent can observe: fieryness, temperature, material and total area (ground area * number of floors).

Data for the regression is extracted from the simulator logs. After several runs on different maps, a significant amount of data from different buildings can be obtained and used as input for the regression.

\subsubsection{Task Allocation}

Fire brigade agents assign a score to each building. The score takes into account the importance of the building (the ones with buried civilians and/or located near gas stations are more important), the distance from the fire brigade to the building and the effort needed to extinguish the fire. Effort estimation takes as input the prediction of how fire evolves (obtained from the regression model previously discussed) and returns a measure of water and time needed to control the fire. The importance of the building increases its score whereas the distance and the effort decreases its score (a fire brigade agent will prefer to extinguish fires on close, important buildings that do not require too much effort).

Figure \ref{fig:firebrigade} shows how the fire brigade agent calculates the score of a building.

\begin{figure}[!ht]
     \centering
     \includegraphics[width=12cm]{img/firebrigade.png}
     \caption{Building score calculation procedure performed by the fire brigade agents.}
     \label{fig:firebrigade}
\end{figure}


After deciding which fire to fight, the fire brigade agent calculates whether it is able to extinguish the fire alone. If not, the agent requests help from other fire brigade agents, using the three-step auction discussed in Section \ref{sec:recruiting}.


%%%SUBSECTION: center
\subsection{Center agents}
\label{sec:center}
In our strategy, center agents listen to radio communication between agents, accumulating knowledge contained in the messages and replicating important messages to avoid loss of information due to communication unreliability. Using the accumulated knowledge from listening to multiple communication channels, center agents are able to build a more accurate world view. Eventually, center agents send messages to agents in specific channels to give them information previously sent in other channels. This is done in order to make knowledge among agents more uniform.

Center agents do not assign tasks to field agents. Their role is merely informational. The decision of engaging in a task is made individually by each field agent. Therefore, our field agents are autonomous decision-makers. The advantage of this approach is that our strategy does not depend heavily on the center agents. Although their knowledge bases will lack some information provided by the center agents, the decision-making process of our field agents will not change, thus they should not be heavily affected by the absence of center agents.

%Nossa abordagem nao depende fortemente de centrais; elas serao usadas apenas para escutar mais de um canal e replicar/propagar mensagens ``importantes''


%\subsection{Common issues}
%\label{sec:common}


% Problemas comuns a todos: temos a compactação de mensagens para a comunicação e o path planning.

%%%SUBSECTION: communication
\subsection{Communication}
\label{sec:communication}

\subsubsection{Message construction}
The goal of communication is to increase the knowledge base of agents and to improve coordination among them. However, in a disaster scenario, communication can be unreliable and bandwidth is limited. To address these problems, our communication strategy is based on a simple protocol. We define three message types for each kind of problem (blockades, burning buildings and buried civilians) the agents may encounter:

\begin{itemize}
 \item Report: the agent found a problem and reports it to its colleagues. Report messages contain information about the problem so that other agents can update their knowledge bases and reason about the recently discovered problem, that is, deciding to engage or not.

 \item Engage: the agent decided to engage in a problem. Engage messages also contain information about the problem so that other agents can update their knowledge bases even if the problem was reported earlier.

 \item Solved: the agent solved the problem and no further action about it is required. Solved messages do not contain information about the problem status (such as the temperature of a building) since the problem was solved and no further action is needed.
\end{itemize}

Coordination among agents, i.e., requesting help for more effort-demanding tasks require a special protocol, described in Section \ref{sec:recruiting}.

In order to save radio bandwidth and maximize the amount of data shared between the agents, our communication will use a data compression algorithm. Each agent is responsible for compressing/decompressing the communication data.

\subsubsection{Recruiting protocol}
\label{sec:recruiting}
When an agent estimates that it needs help to perform a task, it starts an auction which consists of the following three steps: first, a ``request'' message is sent by the recruiter agent. Agents who receive the ``request'' message and are available to help  send back an ``offer'' message. Receiving the ``offer'' messages allow the recruiter agent to select the colleagues that will help best in the task. A ``commit'' message is sent to the selected colleagues. Figure \ref{fig:auction} illustrates the three-step auction.

\begin{figure}[ht]
  \centering
  \subfigure[]{
    {\includegraphics[width=3cm]{img/request.png}}
    \label{fig:request}
  }%
  \subfigure[]{
    {\includegraphics[width=3cm]{img/offer.png}}
    \label{fig:offer}
  }%
  \subfigure[]{
    {\includegraphics[width=3cm]{img/commit.png}}
    \label{fig:commit}
  }%

  \caption{Three-step auction: \subref{fig:request} recruiter (circled with dashed line) sends ``request'' messages; \subref{fig:offer} agents who decide to help respond with ``offer'' messages; \subref{fig:commit} recruiter selects colleagues to help and sends ``commit'' messages.}

 \label{fig:auction}
\end{figure}

It should be noted that messages are always broadcasted in a given channel, that is, all agents can hear all messages in the channel. When it is said that a message is directed to a specific agent, the meaning is that the message is broadcasted in the channel but it has additional information so that non-targets of the message can ignore it.

% Sera implementado um leilão em 3 etapas (anúncio, recebimento de 'ofertas' e recrutamento) para bombeiros pedirem auxílio em tarefas dificeis.
% Compactação


%%%SUBSECTION: path planning
\section{Path planning}
\label{sec:path}

\subsection{Augmented graph}
\label{sec:augraph}
To enhance the basic path planning capabilities provided by the Rescue Simulator's sample agent, we create an augmented neighborhood graph that conveys more information about distance among areas. The augmented graph helps guiding agent navigation and is associated with the Police Force clearing procedures.

In the following discussion, we refer to an area as a building or a road, according to the naming in Rescue Simulator's source code. Besides, two areas are considered neighbors if they are adjacent and an agent can move between them in the absence of blockades.

In the sample search graph (available in the Rescue Simulator package), areas are the nodes and arcs connect neighbor areas. In the augmented graph, a node exists at each door or passage between areas. These nodes are located in the midpoint of the door or passage. Also, an additional node is located at the centroid of each area. Arcs connect all nodes in the same area. 

The augmented graph allows easier and precise computation for its arc weights. The weights are approximated by the euclidean distance between nodes. Weights are more difficult to calculate in the sample search graph as the nodes have no information about position. 
If the area centroid is considered as its location and weight is given by euclidean distance, large errors could occur as the euclidean distance between centroids may diverge greatly from the distance an agent must actually traverse. Manhattan distance would need to consider area orientation or would incur in errors as well. 

Besides more accurate distance calculations, the augmented graph allows more detailed blockade handling: if the path between two adjacent nodes is blocked, we mark only the edge that connects them as impassable. Other edges inside the same area might still be traversable. In the sample search graph, the edges are not associated with positional information, making it difficult to represent detailed blockade information.

%Algorithm \ref{alg:augbuild} shows how the augmented search graph is constructed.
%
%\begin{algorithm}[ht]
%\begin{algorithmic}
%\FORALL{$a \in MapAreas$} 
%\FORALL{$n \in neighbors(a)$} 
%\STATE Insert node at midpoint of $a$-$n$ door/passage
%\ENDFOR
%\STATE Insert node at centroid of $a$
%\ENDFOR
%\end{algorithmic}
%\caption{Augmented graph construction}
%\label{alg:augbuild}
%\end{algorithm}

Figure \ref{fig:graphs} shows sample and augmented graphs. Nodes in the sample search graph have no position al information, but are drawn in area centroid for illustration purposes. %, as they do not convey positional information. 
%The graph is augmented with the squared orange nodes.

\begin{figure}[!ht]
  \centering
  \subfigure[]{
    {\includegraphics[width=5.2cm]{img/graph-sample.png}}
    \label{fig:sample}
  }
  \subfigure[]{
    {\includegraphics[width=5.2cm]{img/graph-aug.png}}
    \label{fig:aug}
  }%
  \caption{Sample \subref{fig:sample} and augmented \subref{fig:aug} search graphs representing a hypothetical region. Roads are in light gray whereas buildings are in dark gray.}
  \label{fig:graphs}
\end{figure}

Each agent has an instance of the augmented graph. During path planning, an extra node is added at the agent's position. This node is connected to all nodes in the same area, with arc weights assigned via euclidean distance. Path planning proceeds by executing Dijkstra's algorithm with origin as the agent's node and destination as the node at the centroid of the destination area. If the agent wants to move to a specific position in the destination area, a node in that position is temporarily added as well.

For police force agents, the path followed during the clearing process is associated with the augmented graph, as the police force agent clears towards the midpoint of the passage between the current and next area in its path. The midpoints between area's passages corresponds to the nodes added to form the augmented graph (i.e., square nodes in Fig. \ref{fig:aug}).

As a measure for robustness, the search falls back to Breadth First Search (BFS) over the sample search graph if the augmented graph cannot be instantiated. 

%\subsection{Static and dynamic path planning}
%\label{sec:static-dynamic}
%
%Path planning is divided in static and dynamic path planning. Static path planning occurs at the pre-processing stage, where we calculate the shortest path between all pairs of map nodes. %This is done by repeated application of Dijkstra's algorithm. 
%Static path planning generates a table with origin-destination pairs and the route between them. During simulation, agents perform a lookup on this table and save processing time to know the shortest path between two nodes. Static path planning works either with BFS over the sample search graph or Dijkstra over the augmented graph.
%
%Dynamic path planning occurs when a blockade is detected in a road. When this happens, the shortest path between the agent's origin and destination is recalculated. With BFS over the sample search graph, the road is marked as blocked whereas with Dijkstra over the augmented graph, the edges that intersect the blockade are marked as blocked. 
%The new route is updated in the agent's origin-destinations table and a message is sent to other agents to warn them about the blockade. Upon receipt of this kind of message, agents mark the routes containing the blocked road or edges as invalid on their origin-destination table. When the blockade is cleared by police forces, a new message is sent so that the agents can agents mark the routes containing the blocked road or edges as valid again.


%%%SECTION: Multi-agent
%\section{Multi-agent competition}
\label{sec:multi-agent}

In the multi-agent competition, the goal is to develop a DCOP algorithm in RMASBench. We choose to implement eXtreme-Ants \citep{Santos&Bazzan2009optmas}. This algorithm follows the extended generalized assignment problem (E-GAP) model of \citep{Scerri+2005} and extends Swarm-GAP \citep{Ferreira+2008ccmms}, a swarm intelligence-inspired task allocation algorithm for dynamic scenarios by explicitly employing a recruiting mechanism.

The E-GAP can be formalized as follows: let $\taskset$ be the set of tasks and $\agentset$ the set of agents. Each agent $i \in \agentset$ has $\agtres{i}$ resources to perform tasks. Each task $j \in \taskset$ consumes $\consumes{i}{j}$ resources of agent $i$, when it performs the task. Each agent $i$ has a capability $\agtcap{i}{j} \in [0,1]$ to perform task $j$. Capability can be regarded as the skill of the agent to perform the task. %A value of $\agtcap{i}{j}$ close to $1$ means that $i$ performs $j$ with high speed or quality.

An allocation matrix $A_{|\agentset| \times |\taskset|}$ has an element $\allocate{i}{j}$ set to $1$ if agent $i$ performs task $j$, and 0 otherwise. In this model, only one agent can perform a given task instance. Thus, a task that would require multiple agents to execute it must be broken down into smaller, inter-related tasks. In the case that all inter-related tasks must be simultaneously allocated, they are related by an AND constraint. 
Formally, let  $\allandtasks \enspace = \{\andtasks{1},\cdots,\andtasks{p} \}$, where $\andtasks{k} = \{j_{k_1},\cdots,j_{k_q}\}$ denotes the $k$-th set of AND-constrained tasks. The partial reward $\partialrwd{i}{j}$ given by the allocation of task $j$ by agent $i$ is defined in Eq. \ref{eq:egap-partial-reward}. The first case refers to an unconstrained task, the second case refers to an AND-constrained task that belongs to a set where all other constrained tasks were allocated and the third case refers to an AND-constrained task that belongs to a set where one or more constrained task was not allocated.

\begin{equation}
\label{eq:egap-partial-reward}
\partialrwd{i}{j} = 
\begin{cases}
  \agtcap{i}{j} \times \allocate{i}{j} ,& \text{if } \forall\andtasks{k} \in \enspace \allandtasks, j \notin \andtasks{k}\\
  \agtcap{i}{j} \times \allocate{i}{j} ,& \text{if } \exists\andtasks{k} \in \enspace \allandtasks \text{ with } j \in \andtasks{k} \wedge \sum_{i \in \agentset}\sum_{j_k \in \andtasks{k}} \allocate{i}{j_k} = |\andtasks{k}| %\forall j_{k_u} \in \alpha_k, a_{xj_{k_u}} \neq 0 
  \\
  0,& \text{otherwise}
\end{cases}
\end{equation}


In E-GAP, the total reward $W$ is calculated as the sum of the rewards of the agents along discrete timesteps. Thus, all variables are indexed by the timestep $t$. In a given timestep, the reward is the sum of partial rewards calculated via Eq. \ref{eq:egap-partial-reward}. A delay cost $d_j^t$ is applied as a penalty for not allocating task $j$ in timestep $t$, as Eq. \ref{eq:egap-total-reward1} illustrates. %The calculation of rewards along the timesteps captures the dynamics of the environment. That is, the reward in a given timestep depends on the tasks and agents that exist in that timestep. 
Equation \ref{eq:egap-total-reward2} determines that agents must allocate tasks within their resource limits and Eq. \ref{eq:egap-total-reward3} determines that a task can be performed by only one agent. %Thus, large tasks must be broken down into smaller tasks that can be performed by a single agent. E-GAP also considers task interdependence, but this aspect will not be investigated in this work. 
%The goal in E-GAP is to maximize the total reward $W$ given by Eq. \ref{eq:egap-total-reward}. Note that terms which are indexed by the timestep $t$ might vary from one timestep to another. For example, new tasks can arise, the resources of the agents might be reduced, etc.
%A Eq. \ref{eq:egap-total-reward} possui semelhança com a Eq. \ref{eq:gap-optimal}, 
\begin{subequations}
	\begin{equation}
	\label{eq:egap-total-reward1}
	W = \sum_t \sum_{i^t \in \agentset^t} \sum_{j^t \in \taskset^t} w_{ij}^t \times a_{ij}^t - 
	\sum_t \sum_{j^t \in \taskset^t}(1 - a_{ij}^t) \times d_j^t 
	\end{equation}
	
	\begin{equation}
	\label{eq:egap-total-reward2}
	\text{ subject to:~~~~~~~~}	
	\forall t \forall i^t \in \agentset^t, \sum_{j^t \in \taskset^t} \consumes{i}{j}^t \times a_{ij}^t \leq \agtres{i}^t 
	\end{equation}
	\begin{equation}
	\label{eq:egap-total-reward3}
	\text{and:~~~~~~~~~}
	\forall t \forall j^t \in \taskset^t, \sum_{i^t \in \agentset^t} a_{ij}^t \leq 1 
	\end{equation}
\end{subequations}

Inspired by the division of labor in social insects, eXtreme-Ants is an approximate algorithm for E-GAP. %In swarms, or colonies of social insects, complex behaviors emerge from the aggregation of individual actions of colony members. %One characteristic of swarms is the ability to respond to changes in the environment by adjusting the numbers of members performing each task.

Observations about swarm behaviors are the base of the model presented in \citep{Theraulaz+1998}, where tasks have associated stimulus %\footnote{Stimulus intensity may be associated with pheromone concentration, the number of encounters with other individuals performing the task or another characteristic that individuals can measure.} 
and individuals have response thresholds for each task. Let $\stimulus{j} \in [0,1]$ be the stimulus associated with task $j \in \taskset$ and $\respthresh{i}{j} \in [0,1]$ be the response threshold of individual (agent) $i$ to task $j$. The tendency, or probability, of individual $i$ to engage in task $j$ is given by $\tendency{i}{j} \in [0,1]$, calculated using Eq. \ref{eq:tendency}.

\begin{equation}
\label{eq:tendency}
\tendency{i}{j} = \frac{\stimulus{j}^2}{\stimulus{j}^2 + \respthresh{i}{j}^2}
\end{equation}

%In swarms, due to polymorphism, individuals may be more able to perform certain kinds of tasks. 
The response threshold \respthresh{i}{j} of individual $i$ to task $j$ is calculated via Eq. \ref{eq:respthresh}, where $\agtcap{i}{j} \in [0,1]$ is the capability of agent $i$ to perform task $j$.

\begin{equation}
\label{eq:respthresh}
\respthresh{i}{j} = 1 - \agtcap{i}{j}
\end{equation}

Regarding independent, i.e., non-constrained tasks, in eXtreme-Ants, agents individually decide which task they will engage in a simple and efficient way, minimizing computational effort and communication between agents. Agents communicate using a token mechanism. When a given agent perceives new tasks, it creates a token with these tasks. The agent can receive tokens from other agents too. Either way, the holder of a token has the right to decide in which tasks of the token it will engage. The token with the remaining tasks is passed to a random agent that has not held the token before. %This is formalized in Alg. \ref{alg:x-ants}, which is executed by each agent independently.

%\begin{algorithm}[ht]
%\begin{algorithmic}
%\STATE When tasks are perceived: $token \gets$ \{perceived tasks\}
%\STATE When message is received: $token \gets$ \{received token\}
%
%\FORALL{task $j \in token$} 
%\STATE{Compute and normalize \tendency{i}{j} for all tasks in $token$}
%\IF{$random() < \tendency{i}{j}$ and $\agtres{i} > c_{ij}$ }
%\STATE{Engage in task $j$}
%\STATE{$token \gets token \setminus \{j\}$}
%\STATE{$\agtres{i} \gets \agtres{i} - c_{ij}$}
%\ENDIF
%\ENDFOR
%
%\STATE Send $token$ to random agent that didn't see the token before
%\end{algorithmic}
%\caption{Non-constrained task monitor}
%\label{alg:x-ants}
%\end{algorithm}

To deal with AND-constraints among tasks, i.e. groups of tasks that demand simultaneous execution, eXtreme-Ants uses a process inspired in the recruitment process for cooperative transport observed among ants. When an agent perceives a set of inter-constrained tasks, it acts as a scout ant. Firstly it attempts to perform all the constrained tasks. If it fails, then it begins a recruitment process via communication. There are five kinds of messages used in the recruitment protocol of eXtreme-Ants. A \msg{request} message is an invitation for an agent to join the recruitment for a task. A \msg{committed} message is sent to inform  that the agent joined the recruitment for a task. A \msg{engage} message informs that the agent was indeed selected to perform the task. A \msg{release} message informs that the agent was not selected to perform the task and must de-commit to it. A \msg{timeout} message informs that the request has expired. Figure \ref{fig:auction} illustrates the recruitment process, except the timeout mechanism. An agent decides to commit to a task probabilistically, via Eq. \ref{eq:tendency}. 

\begin{figure}[ht]
  \centering
  \subfigure[]{
    {\includegraphics[width=3cm]{img/request.png}}
    \label{fig:request}
  }%
  \subfigure[]{
    {\includegraphics[width=3cm]{img/offer.png}}
    \label{fig:offer}
  }%
  \subfigure[]{
    {\includegraphics[width=3cm]{img/commit.png}}
    \label{fig:commit}
  }%

  \caption{Recruitment process: \subref{fig:request} recruiter (circled with dashed line) sends \msg{request} messages; \subref{fig:offer} agents who decide to help respond with \msg{committed} messages; \subref{fig:commit} recruiter sends \msg{engage} to selected teammates and dismisses non-selected ones with \msg{release} messages.}

 \label{fig:auction}
\end{figure}

In addition to the steps presented in Fig. \ref{fig:auction}, in eXtreme-Ants, a \msg{request} message is forwarded when the agent who received it decides to not commit to the task. Besides, there is a mechanism to prevent excessive forwarding of \msg{request} messages: when a \msg{request} message is forwarded too many times, it expires. The agent that received an expired \msg{request} message sends a \msg{timeout} message to the scout (recruiter) agent. The scout, upon receipt of \msg{timeout} messages, sends \msg{release} messages to dismiss agents that had previously committed to the task. 

%The choice of eXtreme-Ants for the multi-agent competition is due to its performance in terms of team reward, computational effort and exchanged messages compared to other E-GAP-based algorithms. Also, it would be interesting to compare its performance with factor-graph based algorithms, such as max-sum, which is implemented in RMASBench \citep{Kleiner+2013}. 

This TDP presents a brief description of eXtreme-Ants. For further details, the reader can refer to \citep{Santos&Bazzan2009optmas}.

%The recruitment process used in the Agent competition (Section \ref{sec:recruiting}) a simplification of the recruitment process of eXtreme-Ants. %Thus, Fig. \ref{fig:auction} also illustrates the recruitment process presented here. 
%In the agent competition we use a simplified version of
%A \textit{timeout} message informs that a request for a task has reached its timeout, and is used to prevent deadlocks. The handling of AND-constrained tasks is formalized in Alg. \ref{alg:x-ants-comm}.

%
%\begin{algorithm}[ht]
%\begin{algorithmic}
%\STATE When set of constrained tasks \andtasks{} is perceived:
%\STATE{$\beta \gets \emptyset$}
%\IF{ $\agtres{i} \geq \sum_{j \in \andtasks{}} \consumes{i}{j}$}
%	\STATE{Compute and normalize \tendency{i}{j} for all tasks in token}
%	\FORALL{task $j \in \andtasks{}$}
%		\IF{$random() < \tendency{i}{j}$ }
%			\STATE{Accept task $j$; $\beta \gets \beta \cup \{j\}$; $\andtasks{} \gets \andtasks{} \setminus \{j\}$}
%		\ENDIF
%		\IF{All tasks in $\andtasks{}$ were accepted, i.e., $\beta = \andtasks{}$}
%			\STATE{Perform all tasks and set $\agtres{i} \gets \agtres{i} - \sum_{j \in \andtasks{}} \consumes{i}{j}$}
%		%\ELSE
%		%	\STATE{Discard all accepted tasks and set: $\andtasks{} \gets \andtasks{} \cup \beta$}
%	    \ENDIF
%    \ENDFOR 
%\ENDIF
%\STATE{recruit($\andtasks{} \setminus \beta$)}
%\end{algorithmic}
%\caption{Constrained task monitor}
%\label{alg:x-ants-comm}
%\end{algorithm}


%\subsection{E-GAP as a DCOP}
%
%In a DCOP, each agent has a variable to which it must assign values corresponding to the tasks it will perform. In the E-GAP, agents may execute multiple tasks at once, thus, DCOP variables may take multiple values simultaneously, as in graph multi-coloring \citep{Scerri+2005}. However, a task cannot be performed by more than one agent. Thus, the same value cannot be assigned to two distinct DCOP variables. This implies a "not-equal" constraint between every agent that can perform the same tasks, which results in dense constraint graphs, which are problematic in DCOP because of the large ammount of commnication required to remove conflicts \citep{Scerri+2005}.
%
%E-GAP-based task allocation algorithms (e.g. LA-DCOP \citep{Scerri+2005}, Swarm-GAP \citep{Ferreira+2008ccmms}, eXtreme Ants \citep{Santos&Bazzan2009optmas}}) use a token mechanism to avoid the communication issue. Tasks are grouped in tokens that are owned by a single agent. Only the token owner may allocate the tasks in it. The token owner may forward it to a teammate. This avoids conflicts and reduces communication.

%A DCOP is a tuple (\variables, \domains, \functions). $\variables = \{x_1,\cdots,x_{|\variables|}\}$ is the set of variables, $\domains = \{D_1,\cdots D_{|\variables|}\}$ is the set of finite and discrete domains, such that each $x_i \in \variables$ takes values in $D_i \in \domains$. Each agent owns a variable and decides which value it will take. \functions~ is the set of functions that define costs of variable assignments. In a DCOP, the goal is to minimize the sum of the costs of functions in \functions. 

%The E-GAP can be formulated as a DCOP as follows:
%
%\begin{itemize}
%%  \item Cada variável $x_i \in V$ representa um agente $i \in \agentset$.
%  \item Let $2^\domains$ be the power set of $\taskset$. Domain $D_i$ of variable $x_i$ is the set of elements in $2^\domains$ such that $\forall d \in D_i, \sum_{j \in d} \rescspt{i}{j} \leq \agtres{i}$. This means that $D_i$ is a set composed of sets of tasks that agent $i$ can perform simultaneously.
%  \item A cost function $f_{kl}$, related to variables $x_k$ e $x_l$ is given by Eq. \ref{eq:dcop-constraint-cost}. The cost (to be minimzed) is the dual of the sum of rewards (to be maximized in E-GAP) obtained by agents $k$ and $l$ via Eq. \ref{eq:egap-partial-reward}. Besides, the function prevents that the same task is allocated by two agents.
%
%\begin{equation}
%\label{eq:dcop-constraint-cost}
%f_{kl} = 
%\begin{cases}
%  - \left(\sum_{j \in D_k} w_{kj} + \sum_{j \in D_l} w_{lj}\right) ,& \text{se } a_{kj} \neq a_{lj}\\
%  \infty,& \text{caso contrário}
%\end{cases}
%\end{equation}
%
%%Uma restrição com o custo dado pela Equação \ref{eq:dcop-constraint-cost} é criada para cada par de variáveis em $V$. 
%\end{itemize}
%The constraint graph of this DCOP formulation is complete, as there is a constraint between each pair of variables. The total number of constraints can be calculated as $\frac{|\variables|(|\variables|-1)}{2}$. The number of constraints grows quadratically with the number of agents (as each agent is represented by a variable). The domain of the variables, on the worst case, where all agents can allocate all tasks is $2^{t}$ where $t$ is the number of tasks. That is, the domain grows exponentially with the number of tasks. This complex DCOP model requires approximate algorithms to be solved.
%
%



%%%SECTION: conclusion
\section{Conclusion}

To address the challenge posed by the agent competition, we present an approach that relies on autonomous decision-making agents. This means that no field or center agent can directly assign a task to other agent. %Each agent decides the task it will engage based on its current knowledge, accumulated from observation and communication. Cooperation emerges when agents decide to engage in the same task, or when help is requested via a recruitment method based which is a simplification eXtreme-Ants recruitment process. Even during recruitment, agents are not directly recruited: they can decide whether to offer help to the recruiter or not. The role of center agents is informational, thus in their absence, the decision-making process of field agents does not change.

Our approach relies on map sectorization in order to limit the scope of action of each field agent. The task allocation of field agents is based on the score that they assign to each task, which is associated with the importance and difficulty level of solving the task. %We believe that the proposed approach is flexible and robust. Flexibility comes from the fact that the task allocation process for all agents is similar: they assign scores to the tasks that they know and engage in the task with the highest score. Robustness comes from the fact that agents do not depend on each other to decide which task to engage. Thus, the approach can work for several configurations of team sizes and presence of center agents.

%For the multi-agent competition, we choose to implement eXtreme-Ants in RMASBench. The algorithm compares favorably to other E-GAP based approaches in terms of team reward, computational effort and exchanged messages. This implementation represents a scientific contribution as we can establish the first comparison of E-GAP and factor graph based approaches (e.g. max-sum already implemented in RMASBench) for task allocation in dynamic environments.% in a single framework.
%Also, it would be interesting to compare its performance with factor-graph based algorithms, such as max-sum, which is implemented in RMASBench \cite{Kleiner+2013}. For further details of eXtreme-Ants implementation and performance, the reader can refer to \cite{Santos&Bazzan2009optmas}.

%falar das estrategias gerais; resultados esperados
% em suma, nossos agentes sao 'autonomos', isto eh, nao recebem ordem direta de outros agentes, mas deliberam por conta propria se vao realizar as tarefas ou nao

%\section*{Acknowledgment}
%\label{sec:ack}


\bibliographystyle{unsrtnat}
%\bibliographystyle{plainnat}
\bibliography{references}

\end{document}
